\documentclass{article}

\usepackage{fancyhdr}
\usepackage{poemscol}

\begin{document}

\poemtitle{139}
\begin{poem}
	\begin{stanza}
    Dezembro dia vinte e um    \verseline
    Jove, o casamento pediu,   \verseline
    Não quero casamento nenhum \verseline
    Seu João, com calma ouviu.
  \end{stanza}
\end{poem}

\poemtitle{140}
\begin{poem}
	\begin{stanza}
    Jove chegou do lar  \verseline
    Disse para Joaquim, \verseline
    Agora é só casar    \verseline
    A resposta foi dada, sim.
  \end{stanza}
\end{poem}

\poemtitle{141}
\begin{poem}
	\begin{stanza}
    Isso era brincadeira   \verseline
    Jove disse, é verdade, \verseline
    Não passa de asneira,  \verseline
    Com toda lealdade.
  \end{stanza}
\end{poem}

\poemtitle{142}
\begin{poem}
	\begin{stanza}
    Dezoito dias depois    \verseline
    O casamento realisava, \verseline
    Agora junto os dois    \verseline
    Joaquim ia trabalhar.
  \end{stanza}
\end{poem}

\poemtitle{143}
\begin{poem}
	\begin{stanza}
    Os dois fizeram viagem      \verseline
    Para a fazenda do Ermelino, \verseline
    Dois cargueiro com bagagem  \verseline
    Levaram um menino.
  \end{stanza}
\end{poem}

\poemtitle{144}
\begin{poem}
	\begin{stanza}
    Joaquim foi faser casa        \verseline
    Para aquele fasendeiro,       \verseline
    Um carpinteiro que não atrasa \verseline
    Precisava ganhar dinheiro.
  \end{stanza}
\end{poem}

\poemtitle{145}
\begin{poem}
	\begin{stanza}
    No fim foram pru'arraial  \verseline
    Com o produto do trabalho \verseline
    Viver não bem e nem mal   \verseline
    Assim quebrando o galho.
  \end{stanza}
\end{poem}

\poemtitle{146}
\begin{poem}
	\begin{stanza}
    **1924**                            \verseline
    Joaquim pelo exercito foi sorteiado \verseline
    Não foi se apresentar               \verseline
    Com mêdo de ser capturado           \verseline
    Andava sempre a meditar.
  \end{stanza}
\end{poem}

\poemtitle{147}
\begin{poem}
	\begin{stanza}
    Nasceu o primeiro filho         \verseline
    Que se chamou Benedito,         \verseline
    A familia alegrou-se com aquilo \verseline
    Foi evento muito bendito.
  \end{stanza}
\end{poem}

\poemtitle{148}
\begin{poem}
	\begin{stanza}
    Joaquim andava a esconder \verseline
    Com mêdo da captura,      \verseline
    Que viesse lhe prender    \verseline
    Lhe dando uma vida dura.
  \end{stanza}
\end{poem}

\poemtitle{149}
\begin{poem}
	\begin{stanza}
    **1925**                    \verseline
    Foi em quatorze de agosto   \verseline
    Vespera da festa da Abadia, \verseline
    Que muito contragosto       \verseline
    Revoltoso no arraial havia.
  \end{stanza}
\end{poem}

\poemtitle{150}
\begin{poem}
	\begin{stanza}
    Joaquim estaba trabalhando \verseline
    Em cerração de madeira,    \verseline
    Olimpia quase chorando     \verseline
    Vamos escoder de carreira.
  \end{stanza}
\end{poem}

\poemtitle{151}
\begin{poem}
	\begin{stanza}
    Joaquim estava no funil     \verseline
    Dando sequencia o trabalho, \verseline
    Ele era forte e varonil     \verseline
    Não tinha nada de atrapalho.
  \end{stanza}
\end{poem}

\poemtitle{152}
\begin{poem}
	\begin{stanza}
    Uma pergunta lhe fiz        \verseline
    O que aconteceu por lá?     \verseline
    Não fui pegada por um triz, \verseline
    A cavalaria lá está.
  \end{stanza}
\end{poem}

\poemtitle{153}
\begin{poem}
	\begin{stanza}
    Correndo iamos esconder   \verseline
    Na fazenda do Vitalino,   \verseline
    La eles não podia nos vêr \verseline
    E, esquecemos do menino.
  \end{stanza}
\end{poem}

\poemtitle{154}
\begin{poem}
	\begin{stanza}
    Ao chegarmos na estrada    \verseline
    Corremos cerrado a fora,   \verseline
    Por causa de uma barulhada \verseline
    Era o festeiro que ia embora.
  \end{stanza}
\end{poem}

\poemtitle{155}
\begin{poem}
	\begin{stanza}
    Escondemos em um mato \verseline
    Depois tive um tino,  \verseline
    Buscar algum boato    \verseline
    Em casa do Quintino.
  \end{stanza}
\end{poem}

\poemtitle{156}
\begin{poem}
	\begin{stanza}
    La encontrei Joaquim dengôso \verseline
    Que vinha de Buriti,         \verseline
    Este logo me perguntou       \verseline
    O que está fasendo aqui?
  \end{stanza}
\end{poem}

\poemtitle{157}
\begin{poem}
	\begin{stanza}
    Tambem lhe perguntei     \verseline
    O que a no arraial?      \verseline
    O povo de Izidorio Lopes \verseline
    Tomando dinheiro, comida e animal.
  \end{stanza}
\end{poem}

\poemtitle{158}
\begin{poem}
	\begin{stanza}
    Vamos esconder           \verseline
    Na cocheira do chafariz, \verseline
    Esperamos amanhã cedo    \verseline
    A ver o que eles diz.
  \end{stanza}
\end{poem}

\poemtitle{159}
\begin{poem}
	\begin{stanza}
    Ao amanhecer do dia    \verseline
    Madamos na povoação,   \verseline
    Espreitar os ocorridos \verseline
    Se podiamos ir ou não.
  \end{stanza}
\end{poem}

\poemtitle{160}
\begin{poem}
	\begin{stanza}
    A resposta foi favoravel    \verseline
    Pelo Juarez Tavora,         \verseline
    No mato não sou responçavel \verseline
    Então vamos embora.
  \end{stanza}
\end{poem}

\poemtitle{161}
\begin{poem}
	\begin{stanza}
    Na frente foi João Alberto \verseline
    La para São Romão,         \verseline
    Por ser revoltoso esperto  \verseline
    Atacar o batalhão.
  \end{stanza}
\end{poem}

\poemtitle{162}
\begin{poem}
	\begin{stanza}
    Coronel Dutra me chamou  \verseline
    Tratei de me esconder,   \verseline
    Era para estrada ensinar \verseline
    Na terceira fui atender.
  \end{stanza}
\end{poem}

\poemtitle{163}
\begin{poem}
	\begin{stanza}
	Cinco dias permaneceram \verseline
	No arraial do Buriti,   \verseline
	No sesto partiram       \verseline
	Má impressão ficou ali.
  \end{stanza}
\end{poem}

\poemtitle{164}
\begin{poem}
	\begin{stanza}
    Passei a noite no acampamento \verseline
    Viajei a cinco horas do dia,  \verseline
    Coronel Dutra bom elemento    \verseline
    Muitas informações me pedia.
  \end{stanza}
\end{poem}

\poemtitle{165}
\begin{poem}
	\begin{stanza}
    Ao completar oitenta e dois    \verseline
    Que me lembro depois           \verseline
    De uma vida tribulada.         \verseline
    Quando na mocidade sobranceira \verseline
    Trabalhando sem canceira       \verseline
    Epoca que me é lembrada.       \verseline
    Eu não posso esquecer          \verseline
    Tanto que recebi sem merecer   \verseline
    Na minha vida conjugal.        \verseline
    Dezoito filhos por excelencia  \verseline
    Sobre a nossa dependencia      \verseline
    Que me livra do mal.           \verseline
    A todos eu agradêço            \verseline
    é o que tenho e oferêço        \verseline
    Sem nenhuma condição.          \verseline
    Pela alegria que tem me dado   \verseline
    Do bom convivio gosado         \verseline
    Suavisando meu coração.
    % A todos neste instante peço  \verseline
    % Nenhum esforço eu meço       \verseline
    % Que continue como são.       \verseline
    % Praticando o espiritismo     \verseline
    % Sem alarde e fanatismo       \verseline
    % Com muito amor e ação.       \verseline
    % Meu agradecimento a Deus     \verseline
    % Por gosos e sofrimentos meus \verseline
    % Neste mundo de provação.     \verseline
    % Que eu não venha me queixar  \verseline
    % Nem tão pouco desertar       \verseline
    % Nesta grande confusão.
  \end{stanza}
\end{poem}

\poemtitle{Completar Oitenta E Dois}
\begin{poem}
  \begin{stanza}
    % Que me lembro depois           \verseline
    % De uma vida tribulada.         \verseline
    % Quando na mocidade sobranceira \verseline
    % Trabalhando sem canceira       \verseline
    % Epoca que me é lembrada.       \verseline
    % Eu não posso esquecer          \verseline
    % Tanto que recebi sem merecer   \verseline
    % Na minha vida conjugal.        \verseline
    % Dezoito filhos por excelencia  \verseline
    % Sobre a nossa dependencia      \verseline
    % Que me livra do mal.           \verseline
    % A todos eu agradêço            \verseline
    % É o que tenho e oferêço        \verseline
    % Sem nenhuma condição.          \verseline
    % Pela alegria que tem me dado   \verseline
    % Do bom convivio gosado         \verseline
    % Suavisando meu coração.        \verseline
    A todos neste instante peço      \verseline
    Nenhum esforço eu meço           \verseline
    Que continue como são.           \verseline
    Praticando o espiritismo         \verseline
    Sem alarde e fanatismo           \verseline
    Com muito amor e ação.           \verseline
    Meu agradecimento a Deus         \verseline
    Por gosos e sofrimentos meus     \verseline
    Neste mundo de provação.         \verseline
    Que eu não venha me queixar      \verseline
    Nem tão pouco desertar           \verseline
    Nesta grande confusão.
  \end{stanza}
\end{poem}

\poemtitle{Sem titulo}
\begin{poem}
  \begin{stanza}
    Erodes, o mandatario         \verseline
    Tinha no Cristo um temerario \verseline
    Que vinha tomar o poder.     \verseline
    Por isso mandou imolar       \verseline
    As crianças do lugar         \verseline
    Para assim livre se ver.     \verseline
    Um espirito avisou José      \verseline
    Aconselhando como é          \verseline
    A fugir para o Egito.        \verseline
    Até que passace a onda       \verseline
    De matança e de ronda        \verseline
    Daquele chefe maldito.
  \end{stanza}
\end{poem}

\poemtitle{Erodes, o mandatario}
\begin{poem}
  \begin{stanza}
    % Tinha no Cristo um temerario   \verseline
    % Que vinha tomar o poder.       \verseline
    % Por isso mandou imolar         \verseline
    % As crianças do lugar           \verseline
    % Para assim livre se ver.       \verseline
    % Um espirito avisou José        \verseline
    % Aconselhando como é            \verseline
    % A fugir para o Egito.          \verseline
    % Até que passace a onda         \verseline
    % De matança e de ronda          \verseline
    % Daquele chefe maldito.         \verseline
    José com seu cajado              \verseline
    E Maria com Cristo ao lado       \verseline
    Montados no jumentinho.          \verseline
    Dimas e Gestas encontraram       \verseline
    Que até encaminharam             \verseline
    Pelo bom e serto caminho.        \verseline
    Os treis no Egito permaneceu     \verseline
    Até que o tempo venceu           \verseline
    O espirito a José avisou.        \verseline
    Agora pode voltar                \verseline
    A gosar de bem estar             \verseline
    O Rei desencarnou.               \verseline
    Em Jeruzalem chegou o Cristo     \verseline
    Que com seus pais era bem quisto \verseline
    Por sua indole amorosa.          \verseline
    la no templo de Salomão          \verseline
    Onde os doutores fazia sermão    \verseline
    Sua palavra poderosa.            \verseline
    Sua falta Maria sentiu           \verseline
    Para onde este menino fugiu      \verseline
    Comessou a procurar.             \verseline
    A discutir na sinagoga           \verseline
    Os doutores lhe enterroga        \verseline
    Qual o mestre a lhe ensinar.
  \end{stanza}
\end{poem}

\end{document}
