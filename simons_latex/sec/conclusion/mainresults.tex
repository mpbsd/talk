\begin{frame}[allowframebreaks]{Main results}
  Let \(M\) be a connected hypersurface immersed with constant mean curvature
  in a space form \(\ambientspace\) of dimension \(n+1\) with constant
  curvature, say, \(c\).

  \framebreak

  \begin{lemma}\label{lemma1}
    If \(M\) is compact and has non-negative sectional curvature (for all
    \(2\)-planes), then we have
    \[
      \immersionconn{A}=0
      \quad\text{and}\quad{\left(\lambda_{i}-\lambda_{j}\right)^{2}K_{ij}=0}
    \]
    for all \(i,j\) at every point of \(M\). In particular, the eigenvalues of
    \(A\) are constant (where the field of unit normals \(\xi\) is defined).
  \end{lemma}

  \framebreak

  \begin{lemma}\label{lemma2}
    If \(M\) has non-negative sectional curvature, and \(f=\trace{A^{2}}\) is
    constant on \(M\), then we have the same conclusions as Lemma~\ref{lemma1}.
  \end{lemma}
  
  \framebreak

  \begin{lemma}\label{lemma3}
    Under the assumptions of Lemma~\ref{lemma1} or Lemma~\ref{lemma2}, either
    \(M\) is totally umbilical or \(A\) has exactly two distinct constants as
    eigenvalues at every point.
  \end{lemma}

  \framebreak

  \begin{theorem}\label{theorem1}
    Let \(M\) be a complete Riemannian manifold of dimension \(n\) with
    non-negative sectional curvature, and \(\phi:M\to\mathbb{R}^{n+1}\) an
    isometric immersion with constant mean curvature into an Euclidean space
    \(\mathbb{R}^{n+1}\). If \(f=\trace{A^{2}}\) is constant on \(M\), then
    \(\phi(M)\) is of the form \(\mathbb{S}^{p}\times\mathbb{R}^{n-p}\),
    \(0\leqslant{p}\leqslant{n}\), where \(\mathbb{R}^{n-p}\) is an
    \((n-p)\)-dimensional subspace of \(\mathbb{R}^{n+1}\), and
    \(\mathbb{S}^{p}\) is a sphere in the Euclidean subspace perpendicular to
    \(\mathbb{R}^{n-p}\). Except for the case \(p=1\), \(\phi\) is an
    imbedding.
  \end{theorem}

  \framebreak

  \begin{corollary}\label{corollary1oftheorem1}
    If \(M\) is, in particular, minimal in Theorem~\ref{theorem1}, then
    \(\phi(M)\) is a hyperplane and \(\phi\) is an imbedding.
  \end{corollary}

  \framebreak

  \begin{corollary}\label{corollary2oftheorem1}
    Let \(M\) be a connected compact Riemannian manifold of dimension \(n\)
    with non-negative sectional curvature. If \(\phi:M\to\mathbb{R}^{n+1}\) is
    an isometric immersion with constant mean curvature, then \(\phi(M)\) is a
    hypersphere and \(\phi\) is an imbedding.
  \end{corollary}

  \framebreak

  \begin{theorem}\label{theorem2}
    Let \(M\) be an \(n\)-dimensional complete Riemannian manifold with
    non-negative sectional curvature, and \(\phi:M\to\mathbb{S}^{n+1}\) an
    isometric immersion with constant mean curvature. If \(f=\trace{A^{2}}\) is
    constant on \(M\), then either
    \begin{enumerate}
      \item
        \(\phi(M)\) is a great or small sphere in \(\mathbb{S}^{n+1}\), and
        \(\phi\) is an imbedding; or
      \item
        \(\phi(M)\) is a product \(\mathbb{S}^{p}(r)\times\mathbb{S}^{q}(s)\),
        and for \(p\neq{1,n-1}\), \(\phi\) is an imbedding.
    \end{enumerate}
  \end{theorem}

  \framebreak

  \begin{corollary}\label{corollary1oftheorem2}
    If \(M\) is, in particular, minimal in Theorem~\ref{theorem2}, then
    \(\phi(M)\) is a great sphere or
    \(\mathbb{S}^{p}\left(\sqrt{p/n}\right)\times\mathbb{S}^{n-p}\left(\sqrt{(n-p)/n}\right)\).
  \end{corollary}

  \framebreak

  \begin{corollary}\label{corollary3oftheorem2}
    Let \(M\) be a connected compact Riemannian manifold of dimension \(n\)
    with non-negative sectional curvature. If \(\phi:M\to\mathbb{S}^{n+1}\) is
    an isometric immersion with constant mean curvature, then the conclusion of
    Theorem~\ref{theorem2} holds.
  \end{corollary}
  
  \framebreak

  \begin{corollary}\label{corollary3oftheorem2}
    Let \(M\) be a connected compact minimal hypersurface immersed in
    \(\mathbb{S}^{n+1}\). If \(M\) has positive sectional curvature, then \(M\)
    is imbedded as a great sphere.
  \end{corollary}

\end{frame}
