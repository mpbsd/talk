\begin{frame}[allowframebreaks]{Computing the Laplacian of \(f=\trace{A^{2}}\)}
  \begin{align*}
    Yf & = Y\left(\trace{A^{2}}\right)                                   \\
       & = \trace{\immersionconn_{Y}A^{2}}                               \\
       & = \trace{(\immersionconn_{Y}A)A+A(\immersionconn_{Y}A)}         \\
       & = \trace{(\immersionconn_{Y}A)A}+\trace{A(\immersionconn_{Y}A)} \\
       & = 2\trace{\left(\immersionconn_{Y}A\right)A}
  \end{align*}

  \framebreak

  \begin{align*}
    XYf                               & = X\left(2\trace{\left(\immersionconn_{Y}A\right)A}\right)                                                                               \\
                                      & = 2\trace{\immersionconn_{X}\left(\immersionconn_{Y}A\right)A+\left(\immersionconn_{Y}A\right)\left(\immersionconn_{X}A\right)}          \\
                                      & = 2\trace{\immersionconn_{X}\left(\immersionconn_{Y}A\right)A}+2\trace{\left(\immersionconn_{Y}A\right)\left(\immersionconn_{X}A\right)} \\
                                      &                                                                                                                                          \\
    \left(\immersionconn_{X}Y\right)A & = 2\trace{\left(\immersionconn_{\immersionconn_{X}Y}A\right)A}                                                                           \\
                                      &                                                                                                                                          \\
    H_{f}(X,Y)                        & = XYf-\immersionconn_{\immersionconn_{X}Y}f                                                                                              \\
                                      & = 2\trace{\left(\immersionconn_{X}\left(\immersionconn_{Y}A\right)-\immersionconn_{\immersionconn_{X}Y}A\right)A}                        \\
                                      & \qquad+2\trace{\left(\immersionconn_{Y}A\right)\left(\immersionconn_{X}A\right)}                                                         \\
                                      & = 2\trace{\left(\immersionconn^{2}A(;Y;X)\right)A}+2\trace{\left(\immersionconn_{Y}A\right)\left(\immersionconn_{X}A\right)}
  \end{align*}

  \framebreak

  Hence, we get that

  \begin{align*}
    \frac{1}{2}\laplacian{f} & = \sum_{i=1}^{n}\left\{\trace{\left(\immersionconn^{2}A(;e_{i};e_{i})\right)A}+\trace{\left(\immersionconn_{e_{i}}A\right)^{2}}\right\} \\
                             & = \trace{\left(\restrictedlaplacian{A}\right)A}+\sum_{i=1}^{n}\trace{\left(\immersionconn_{e_{i}}A\right)^{2}}                          \\
  \end{align*}

  \framebreak

  And by extending the metric \(g\) to the tensor space in the standard way, we
  may write

  \begin{equation}\label{eq:laplacian-of-f-in-terms-of-the-restricted-laplacian-of-A}
    \frac{1}{2}\laplacian{f}=g(\restrictedlaplacian{A},A)+g(\immersionconn{A},\immersionconn{A})
  \end{equation}

  \framebreak

  We shall now compute \(\restrictedlaplacian{A}\). For this purpose, let us write

  \begin{align*}
    \ShapeOperatorsSecondCovariantDerivative(X,Y) & =\immersionconn^{2}A(;Y;X)                                                                \\
                                                  & =\immersionconn_{X}\left(\immersionconn_{Y}A\right)-\immersionconn_{\immersionconn_{X}Y}A
  \end{align*}

  \framebreak

  Using the identities

  \[
    \immersionconn_{X}Y-\immersionconn_{Y}X-[X,Y]=0,
  \]

  and

  \[
    R(X,Y)=[\immersionconn_{X},\immersionconn_{Y}]-\immersionconn_{[X,Y]},
  \]

  where the curvature transformation \(R(X,Y)\) and the other term ares
  regarded as derivations of the algebra of tensor fields, we obtain

  \begin{equation}\label{eq:symmetry-of-the-second-covariant-derivative-of-the-shape-operator}
    \ShapeOperatorsSecondCovariantDerivative(X,Y)=\ShapeOperatorsSecondCovariantDerivative(Y,X)+[R(X,Y),A].
  \end{equation}

  \framebreak

  Let \(\left\{e_{1},\ldots,e_{n}\right\}\) be an orthonormal basis in
  \(\tangentspace{M}{x}\), and extend them to vector fields
  \(E_{1},\ldots,E_{n}\) in a neighborhood of \(x\) such that
  \(\immersionconn{E_{i}}=0\) at \(x\). Let \(X\) be a vector field such that
  \(\immersionconn{X}=0\) at \(x\).

  \framebreak

  \begin{align*}
    \ShapeOperatorsSecondCovariantDerivative(E_{i},X)E_{i} & =\left(\immersionconn_{E_{i}}\left(\immersionconn_{X}A\right)\right)E_{i}-\left(\immersionconn_{\immersionconn_{E_{i}}X}A\right)E_{i}              \\
                                                           & =\immersionconn_{E_{i}}\left(\left(\immersionconn_{X}A\right)E_{i}\right)-\left(\immersionconn_{X}A\right)\left(\immersionconn_{E_{i}}E_{i}\right) \\
                                                           & =\immersionconn_{E_{i}}\left(\left(\immersionconn_{E_{i}}A\right)X\right)                                                                          \\
                                                           & =\left(\immersionconn_{E_{i}}\left(\immersionconn_{E_{i}}A\right)\right)X+\left(\immersionconn_{E_{i}}A\right)\left(\immersionconn_{E_{i}}X\right) \\
                                                           & =\left(\immersionconn_{E_{i}}\left(\immersionconn_{E_{i}}A\right)\right)X-\left(\immersionconn_{\immersionconn_{E_{i}}E_{i}}A\right)X              \\
                                                           & =\ShapeOperatorsSecondCovariantDerivative(E_{i},E_{i})X
  \end{align*}

  \framebreak

  We then get that

  \begin{equation}\label{eq:shape-operators-second-convariant-derivative-at-ei-ei}
    \ShapeOperatorsSecondCovariantDerivative(E_{i},E_{i})X=\ShapeOperatorsSecondCovariantDerivative(X,E_{i})E_{i}+[R(E_{i},X),A]E_{i}
  \end{equation}

  at \(x\).

  \framebreak

  Similarly, we get that

  \begin{equation}\label{eq:shape-operators-second-covariant-derivative-at-x-ei}
    \ShapeOperatorsSecondCovariantDerivative(X,E_{i})E_{i}=\immersionconn_{X}\left(\left(\immersionconn_{E_{i}}A\right)E_{i}\right)
  \end{equation}

  also at \(x\).

  \framebreak

  From now on we assume that \(M\) has \textit{constant mean curvature}, that
  is, \(\trace{A}=\text{constant}\). Under this assumption, we prove that

  \begin{equation}\label{eq:shape-operator-first-covariant-derivative-null-sum-property}
    \sum_{i=1}^{n}\left(\immersionconn_{E_{i}}A\right)E_{i}=0
  \end{equation}

  \framebreak

  Indeed, we have that

  \begin{align*}
    g\left(\sum_{i=1}^{n}\left(\immersionconn_{E_{i}}A\right)E_{i},Z\right) & =\sum_{i=1}^{n}g\left(E_{i},\left(\immersionconn_{E_{i}}A\right)Z\right) \\
                                                                            & =\sum_{i=1}^{n}g\left(E_{i},\left(\immersionconn_{Z}A\right)E_{i}\right) \\
                                                                            & =\trace{\immersionconn_{Z}A}                                             \\
                                                                            & =Z\left(\trace{A}\right) = 0
  \end{align*}

  Since this is valid for an arbitrary vector field \(Z\), we
  conclude~\eqref{eq:shape-operator-first-covariant-derivative-null-sum-property}.

  \framebreak

  Now we have that

  \begin{equation}\label{eq:shape-operator-second-covariant-derivative-null-sum-property}
    \sum_{i=1}^{n}\ShapeOperatorsSecondCovariantDerivative(X,E_{i})E_{i}
    =\immersionconn_{X}\left(\sum_{i=1}^{n}\left(\immersionconn_{E_{i}}A\right)E_{i}\right)=0
  \end{equation}
  
  From~\eqref{eq:shape-operators-second-convariant-derivative-at-ei-ei}
  and~\eqref{eq:shape-operator-second-covariant-derivative-null-sum-property}
  we get

  \framebreak 

  \begin{equation}\label{eq:formula-for-the-restricted-laplacian-of-A}
    \begin{split}
      \left(\restrictedlaplacian{A}\right)(X) & = \sum_{i=1}^{n}\left(\immersionconn^{2}A\right)(X;E_{i};E_{i})                                                                                                    \\
                                              & =\sum_{i=1}^{n}\left\{\left(\immersionconn_{E_{i}}\left(\immersionconn_{E_{i}}A\right)\right)X-\left(\immersionconn_{\immersionconn_{E_{i}}E_{i}}A\right)X\right\} \\
                                              & =\sum_{i=1}^{n}\ShapeOperatorsSecondCovariantDerivative(E_{i},E_{i})X                                                                                              \\
                                              & =\sum_{i=1}^{n}\left\{\ShapeOperatorsSecondCovariantDerivative(X,E_{i})E_{i}+[R(E_{i},X),A]E_{i}\right\}                                                           \\
                                              & =\sum_{i=1}^{n}[R(E_{i},X),A]E_{i}
    \end{split}
  \end{equation}

  \framebreak

  By the Gauss equation~\eqref{eq:gauss} we have

  \[
    R(E_{i},X)=cE_{i}\wedge{X}+AE_{i}\wedge{AX}.
  \]

  Hence

  \framebreak

  \begin{align*}
    \sum_{i=1}^{n}R(E_{i},X)AE_{i} & =\sum_{i=1}^{n}c\left\{g(AE_{i},X)E_{i}-g(E_{i},AE_{i})X\right\}         \\
                                   & \quad+\sum_{i=1}^{n}\left\{g(AE_{i},AX)AE_{i}-g(AE_{i},AE_{i})AX\right\} \\
                                   & =cAX-c\trace{A}X+A^{3}X-\trace{A^{2}}AX
  \end{align*}

  Similarly, we get that

  \[
    \sum_{i=1}^{n}AR(E_{i},X)E_{i}=cAX-cnAX+A^{3}-\trace{A}A^{2}X
  \]

  \framebreak

  \begin{align*}
    \sum_{i=1}^{n}[R(E_{i},X),A]E_{i} & =ncAX-\trace{A^{2}}AX             \\
                                      & \quad-c\trace{A}X+\trace{A}A^{2}X
  \end{align*}

  \framebreak

  \begin{equation}\label{eq:restricted-laplacian-of-the-shape-operator}
    \restrictedlaplacian{A}=ncA-\trace{A^{2}}A-c\trace{A}I+\trace{A}A^{2},
  \end{equation}
  
  where \(I\) is the identity operator.
  From~\eqref{eq:laplacian-of-f-in-terms-of-the-restricted-laplacian-of-A}, we
  obtain

  \begin{equation}\label{eq:final-form-for-the-laplacian-of-f}
    \begin{split}
      \frac{1}{2}\laplacian{f} & =cn\trace{A^{2}}-\left(\trace{A^{2}}\right)^{2}-c\left(\trace{A}\right)^{2}   \\
                               & \quad+\trace{A}\trace{A^{3}}+g\left(\immersionconn{A},\immersionconn{A}\right)
    \end{split}
  \end{equation}

  \framebreak

  In particular, if \(M\) is minimal in \(\ambientspace\), that is, if
  \(\trace{A}=0\), then

  \[
    \restrictedlaplacian{A}=ncA-\trace{A^{2}}A
  \]

  and

  \[
    \frac{1}{2}\laplacian{f}=cnf-f^{2}+g\left(\immersionconn{A},\immersionconn{A}\right)
  \]
  
  \framebreak

  \begin{lemma}
    Let \(A\) be an \(n\times{n}\) symmetric matrix with eigenvalues
    \(\lambda_{1},\ldots,\lambda_{n}\). Then, for any constant \(c\),
    we have that
    \begin{align*}
      nc\trace{A^{2}}&-\left(\trace{A^{2}}\right)^{2}-c\trace{A}^{2}                                                                  \\
                     & +\trace{A}\trace{A^{3}}=\sum_{i<j}\left(\lambda_{i}-\lambda_{j}\right)^{2}\left(c+\lambda_{i}\lambda_{j}\right)
    \end{align*}
  \end{lemma}
  
  \framebreak

  Now, for each point \(x\) of the hypersurface \(M\), let
  \(\left\{e_{1},\ldots,e_{n}\right\}\) be an orthonormal basis in
  \(\tangentspace{M}{x}\) such that

  \[
    Ae_{i}=\lambda_{i}e_{i}\quad{(1\leqslant{i}\leqslant{n})}.
  \]

  Then, equation~\eqref{eq:final-form-for-the-laplacian-of-f} can be written as
  follows

  \begin{equation}\label{eq:formula-for-the-laplacian-of-f-containing-the-sectional-curvature-of-the-immersion}
    \frac{1}{2}\laplacian{f}
    =\sum_{i<j}\left(\lambda_{i}-\lambda_{j}\right)^{2}K_{ij}+g\left(\immersionconn{A},\immersionconn{A}\right)
  \end{equation}
  
\end{frame}
