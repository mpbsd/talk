\begin{frame}[allowframebreaks]{Notation and basic facts}
	Let \(\ambientspace\) be an \((n+1)\)-dimensional space form, that is, a
	Riemannian manifold of dimension \(n+1\) and constant sectional curvature,
	let's say, \(c\in\left\{-1,0,1\right\}\).

	\framebreak

	Let \(\phi:M\to\ambientspace\) be an isometric immersion of an
	\(n\)-dimensional Riemannian manifold \(M\) into \(\ambientspace\).

	\framebreak

	For simplicity, we say that \(M\) is a hypersurface immersed into
	\(\ambientspace\) and, for all local formulas and computations, we may regard
	\(\phi\) as an imbedding and thus identify \(x\in{M}\) with
	\(\phi(x)\in\ambientspace\).

	\framebreak

	The tangent space \(\tangentspace{M}{x}\) is identified with a subsapce of
	the tangent space \(\tangentspace{\ambientspace}{x}\), and the normal space
	\(\normalspace{M}{x}\) is the subspace of \(\tangentspace{\ambientspace}{x}\)
	consiting of all \(X\in\tangentspace{\ambientspace}{x}\) which are orthogonal
	to \(\tangentspace{M}{x}\) with respect to the Riemannian metric \(g\).

	\framebreak

	For an arbitrary point \(x_{0}\in{M}\), there exists an open neighborhood
	\(U\) of \(x_{0}\) in \(\ambientspace\) and a field of unit vectors
	\[
		\xi:U\to{T\ambientspace},
	\]
	normal to all of \(U\cap{M}\).

	\framebreak

	The second fundamental form \(h\) and the corresponding symmetric operator
	\(A\) are defined and related to the covariant differentiations
	\(\ambientconn\) of \(\ambientspace\) and \(\immersionconn\) of
	\(M\) by the following formulas:

	\begin{eqnarray}
		\ambientconn_{X}Y   & = & \immersionconn_{X}Y+h(X,Y),\label{eq:second-fundamental-form}\\
		\ambientconn_{X}\xi & = & -AX,\label{eq:shape-operator}
	\end{eqnarray}

	where \(X\) and \(Y\) are (local) vector fields tangent to \(M\).

	\framebreak

	We then have the equations of Gauss
	\begin{equation}\label{eq:gauss}
		\forevery{X,Y}\in\tangentspace{M}{x}:\quad{R(X,Y)=cX\wedge{Y}+AX\wedge{AY}},
	\end{equation}
	and Codazzi
	\begin{equation}\label{eq:codazzi}
		\forevery{X,Y}\in\tangentspace{M}{x}:
		\quad{\tensorcovariantderivative{A}{\immersionconn}{X}{(Y)}=\tensorcovariantderivative{A}{\immersionconn}{Y}{(X)}},
	\end{equation}
	where
	\[
		\forevery{X,Y,Z}\in\tangentspace{M}{x}:\quad{(X\wedge{Y})Z=g(Y,Z)X-g(X,Z)Y}.
	\]

	\framebreak

	Since \(\xi\) is defined locally up to a sign, so is \(A\). Thus, \(A^{2}\)
	is globally defined.

	\framebreak

	We wish to compute the Laplacian of the function
	\[
		f=\trace{A^{2}}\in\functionspace{M}.
	\]
	This is given as the trace of the symmetric bilinear for

	\begin{equation}\label{eq:hessian-of-f}
		H_{f}(X,Y)=X(Yf)-(\immersionconn_{X}Y)f.
	\end{equation}

	If \(\left\{e_{1},\ldots,e_{n}\right\}\) is an orthonormal basis of
	\(\tangentspace{M}{x}\), then

	\begin{equation}\label{eq:laplacian-of-f-in-an-orthonormal-basis}
		(\laplacian{f})(x)=\sum_{i}H_{f}(e_{i},e_{i}).
	\end{equation}

	\framebreak

	If \(T\) is a tensor field of type \((r,s)\) on \(M\), then the second
	covariant derivative \(\immersionconn^{2}T\) is given by the formula

	\begin{equation}\label{eq:second-covariant-derivative-of-tensor-fields}
		\left(\immersionconn^{2}T\right)(;Y;X)=\immersionconn_{X}\left(\immersionconn_{Y}T\right)-\immersionconn_{\immersionconn_{X}{Y}}T,
	\end{equation}

	where \(X\) and \(Y\) are vector fields on \(M\).

	\framebreak

	At each point \(x\in{M}\), we take an orthonormal basis
	\(\left\{e_{1},\ldots,e_{n}\right\}\) in \(\tangentspace{M}{x}\) and set

	\begin{equation}\label{eq:restricted-laplacian-of-tensor-fields}
		(\restrictedlaplacian{T})(x)=\sum_{i=1}^{n}\left(\immersionconn^{2}T\right)(;e_{i};e_{i}).
	\end{equation}

	This is independent of the choice of an orthonormal basis and the tensor
	field \(\restrictedlaplacian{T}\) so defined, which is again of type
	\((r,s)\), is called the \textit{restricted Laplacian} of \(T\).

  \framebreak

  When \(T\) is a function \(f\), then \(\immersionconn^{2}T\) coincides with
  \(H_{f}\) and \(\restrictedlaplacian{T}\) is nothing but \(\laplacian{f}\).

  \framebreak

  The expression for \(\restrictedlaplacian{T}\) in conventional tensor notation is
  \[
    \left(\restrictedlaplacian{T}\right)^{i_{1}\cdots{i_{r}}}_{j_{1}\cdots{j_{s}}}
    =\sum_{a,b=1}^{n}g^{ab}T^{i_{1}\cdots{i_{r}}}_{j_{1}\cdots{j_{s}};a;b}
  \] 
\end{frame}
